\documentclass[10pt,twocolumn,letterpaper]{article}


% This article is written, editd, and reviewd by the Activeeon R&D department
% for the H2020 MORPHEMIC project. The mentioned work is the intellectual
% property of MORPHEMIC.
% Corresponding author: Ali J. Fahs


% packages
\usepackage[spanish,english]{babel}
\usepackage[utf8x]{inputenc}
\usepackage[T1]{fontenc}
\usepackage[a4paper,top=3cm,bottom=2cm,left=3cm,right=3cm,marginparwidth=1.75cm]{geometry}
\usepackage{amsmath}
\usepackage{graphicx}
\usepackage[colorinlistoftodos]{todonotes}
\usepackage[colorlinks=true, allcolors=blue]{hyperref}
\usepackage[none]{hyphenat}

\title{
		\usefont{OT1}{bch}{b}{n}
		\huge MORPHEMIC: a complete paradigm from Cloud to Edge
}

\usepackage{authblk}

\author[1]{Ali J. Fahs}
\author[1]{Maroun Koussaifi}
\author[1]{Mohamed Khalil Labidi}


	\affil{Activeeon, Research \& Development Department}

\begin{document}
\maketitle

\selectlanguage{english}

{\textbf{Keywords} Edge computing, Cloud computing, MORPHEMIC}

\vskip 2em
\textbf{Cloud limitations and how the Edge can address them}
\vskip .5em

Cloud computing has revolutionized the IT industry by shifting from
traditional in-house servers toward Cloud-based services. A survey in 2019 has
shown that 94\% of technical professionals across a broad cross-section of
organizations use cloud one way or another~\cite{Flexera}. Out of the 786
surveyed professionals, 84\% are using a Multi-Cloud strategy. This can be
attributed to the low cost of obtaining and maintaining the resources and the
facility of scaling the resources following a surge in traffic.

However, the upsurge in fields like Artificial Intelligence (AI), Virtual
Reality (VR), stream processing, autonomous vehicles, and most prominently
Internet of things (IoT) have introduced new requirements to those of the
traditional client-server applications~\cite{ahmed2019fog}. Such requirements
are constricted in terms of:

\begin{itemize}

\item \textbf{Latency:} for example, some VR and gaming applications can only
tolerate end-to-end response times of 20 ms (including network and computation
delays).

\item \textbf{Bandwidth:} for example, stream processing and IoT applications,
requiring the transfer of large data to the Cloud to be processed.

\end{itemize}

In contrast, Cloud computing is constrained by a relatively high network
latency that can be estimated as 40 ms for wired connection and 150 ms for a
4G connection~\cite{claudit}. As a result, latency-sensitive applications like
VR and gaming will perform poorly on the Cloud~\cite{elbamby2018toward}. In
addition, the data transfer from the user to the Cloud is limited by the
Internet Service Provider (ISP) Bandwidth which contributes to impeding
data-intensive applications like stream processing and IoT.

Unlike Cloud computing where resources are centralized in a handful of data
centers, Edge computing aims at distributing resources in the vicinity of the
end-user~\cite{sill2017standards}. This approach guarantees a low
user-to-resource latency that can be as low as a couple of
milliseconds~\cite{fahs2020proximity,fahs2020voila}. Simultaneously, the
end-user is typically connected to the application resources using a Local
Area Network (LAN), and as a result, does not suffer an ISP bandwidth
limitation~\cite{bonomi2012fog}.

\vskip 2em
\textbf{MORPHEMIC combines the best of both worlds}
\vskip .5em

The MORPHEMIC project supports the accumulation of computational nodes from
versatile resource pools. MORPHEMIC does not only provide a Multi-Cloud
solution but also allows application developers to acquire Edge nodes. This
paradigm combines the best of both worlds; resource-intensive applications can
profit from a Multi-Cloud implementation. At the same time, latency-sensitive
applications can deploy their services on the Edge through the same platform.
This opens the door for multitude of optimization possibilities,using
MORPHEMIC, an application can be deployed in a hybrid fashion where the
latency- and bandwidth-critical components of the application are deployed on
the Edge. While the latency-tolerant, data-intensive, or resource-intensive
are deployed components on the Cloud.

This optimization is the core of the MELODIC Uppperware, a component capable
of analyzing the application model and deciding what nodes to be provisioned
based on each application's component requirements. This process is then
enhanced through monitoring and forecasting features provided in the MORPHEMIC
pre-processor~\cite{d41}, such that the application resources are not only
decided at the initial implementation but also subject to on-the-fly
modifications.


\vskip 2em
\textbf{Develop your Edge application and MORPHEMIC will handle the
resource management}
\vskip .5em

MORPHEMIC uses the state-of-the-art ProActive Scheduler and Resource Manager
to handle the resources~\cite{d41,d53}. ProActive a product of Activeeon is a
Java-based cross-platform workflow scheduler and resource manager. ProActive
acts as a central point that connects MORPHEMIC with all Cloud providers,
clusters, and on-premises computational nodes.

ProActive supports the integration of Edge nodes using two main methods:

\begin{itemize}

\item \textbf{BYON infrastructure: } i.e. Bring your own nodes, which supports
any private node provided by the developer. This approach works using an SSH
connection, then ProActive creates and connect an abstract node using a
ProActive node agent. This approach supports any kind of node (bare-metal or
virtual machine) regardless of the sources of the machine (Cloud, Edge,
on-premises, cluster, etc.).

\item \textbf{Edge infrastructure: } this approach targets Edge nodes and
focuses on the stringent resource limitation the Edge nodes are famous for.
Similar to BYON, it deploys a node using a ProActive node agent, however with
the focus on lightweight implementation.

\end{itemize}

As can be seen from the current state-of-the-art, ARMv7 and more specifically
Raspberry Pis are commonly used as the main-stream Edge nodes. MORPHEMIC and
ProActive have taken notice of this trend and enhanced their resource
management to support such devices in terms of supporting ARMv7 and by coping
with their extreme resource limitations. 


\vskip 2em
\textbf{MORPHEMIC adapts resources according to your application needs}
\vskip .5em

In conclusion, supporting different resources types is the first building
blocking of MORPHEMIC. Using this capability MORPHEMIC is able to select the
resources that deliver the best performance to the application's specific
needs. Equally, MORPHEMIC can leverage this capability to deliver a tailored
proactive adaptation based on the application's traffic.


\bibliographystyle{IEEEtran}
\bibliography{MOR.bib}

\end{document}
